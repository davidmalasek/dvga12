\documentclass[12pt, a4paper, openright]{article}

\usepackage[utf8]{inputenc}
\usepackage[english]{babel}
\usepackage{hyperref}
\usepackage[T1]{fontenc}
\usepackage{caption}
\usepackage{listings}
\usepackage{appendix}
\usepackage{graphicx}

\usepackage{ifxetex}
\ifxetex
   \usepackage{fontspec}
   \setmainfont{Georgia}
\else
   \usepackage{times}
\fi

\lstset{breakatwhitespace=false,
  breaklines=true,
  captionpos=b,
  basicstyle=\ttfamily\small
}

% --- Change these three lines as needed ---
\title{Exercise 3 Report}
\author{David Malášek}
\date{\today}

% --- Add custom fields for birth date and statement ---
\newcommand{\birthdate}{13 May 2004}
\newcommand{\affirmation}{I hereby affirm that this report represents my own independent work and that I have neither given nor received unauthorized assistance in its completion.}

\makeatletter
\renewcommand{\maketitle}{%
  \bgroup\setlength{\parindent}{0pt}
  \begin{flushleft}

  \vspace*{-1cm}
  \noindent\includegraphics[width=3cm]{kaulogo.jpg}
  \vspace{2cm}

  \hrule\vspace{0.5cm}
  \textbf{\Huge\@title}
  \vspace{0.5cm}\hrule
  
  \vspace{0.5cm}
  \@author \\
  Born: \birthdate
  
  \vspace{0.5cm}
  \@date
  
  \vspace{0.5cm}
  DVGA12 Programming and Data Structures
  
  Computer Science
  
  Faculty of Health, Science and Technology
  
  \vspace{0.5cm}\hrule

  \vspace{1.5cm}
  \textit{\affirmation}

  \end{flushleft}
  \egroup
}
\makeatother


\begin{document}

\maketitle  % Print title, author, date

\newpage

\tableofcontents  % Print table of contents

\newpage

\section{Introduction}

This report describes the program of Exercise 3 written in C. It simulates a car registry, that can store up to 10 vehicles, along with their owners. User can manipulate with the registry through text based user interface. The main purpose of the program is clearly for students to practice file handling, safe input, usage of structs and arrays, error handling, modularity and the use of Makefile.

\section{Overview}

As said, the user can manipulate the car registry through text based user interface. He has exactly 7 different options, 8th options quits the program. Each of the options is listed below along with a brief description:

\begin{enumerate}
  \item \textbf{Add vehicle} \\
  User is prompted for type, brand, license plate, owner name and owner age. Record consisted of this information will be added to the registry.

  \item \textbf{Remove vehicle} \\
  User enters an index of a vehicle in the registry that should be removed. Respective vehicle will be removed.

  \item \textbf{Sort vehicles} \\
  Vehicles in the registry will be sorted alphabetically based on the owner name.

  \item \textbf{Info about vehicle} \\
  User will enter an index of a vehicle, information about this vehicle only will be printed.

  \item \textbf{Show all vehicles} \\
  Program will print information of all the vehicles in the registry.

  \item \textbf{Add random vehicle} \\
  Program will randomly choose from 5 lists of demo data and create a vehicle which then adds to the registry.

  \item \textbf{Search vehicle} \\
  User will be prompted to enter a owner's name. This query will then be used to search for a vehicle with such owner's name, and the first occurence will be printed.

  \item \textbf{Quit} \\
  The program will terminate.
\end{enumerate}

The program also prints error or warning messages based on the situation.

\section{Detailed Description}

\subsection*{Data Structures}

Program uses 2 structs, \texttt{vehicle} and \texttt{person}. \\
Struct \texttt{vehicle} includes 3 string properties (\texttt{type}, \texttt{brand}, \texttt{license\_plate}) 
and a \texttt{person} property, which links to the second struct \texttt{person}, 
which includes a string property \texttt{name} and an integer property \texttt{age}.

\begin{figure}[h!]
    \centering
    \fbox{\includegraphics[width=0.5\linewidth]{structs.png}}
    \caption{Used structs}
    \label{fig:placeholder}
\end{figure}

\subsection*{Registry}

Registry in the prespective of storage is a plain CSV file. Each value is separated by a comma, each record on it's own line.

\begin{figure}[h!]
    \centering
    \fbox{\includegraphics[width=0.5\linewidth]{registry.png}}
    \caption{Registry}
    \label{fig:placeholder}
\end{figure}

\subsection*{Main functionalities}

Here is a detailed description of logic behind each option.

\begin{enumerate}
  \item \textbf{Add vehicle} \\

  First, function checks if the registry is full. If yes, the execution ends here. Then, validation for each input begins. It prompts the user to enter specific value (eg. vehicle type) and then proceeds to read the input using \texttt{fgets()} function. \\

  The function first checks, if the input was read succesfully, and then proceeds to 2 step validation process. First it checks, if the value is withing the set limit. String values have character amount limit, integers have lower and upper limit. Secondly, it checks if string value is valid in terms of allowed characters. \\

  Here is an example of retrieving and validation for \texttt{type} value:
    
    \begin{figure}[h!]
        \centering
        \fbox{\includegraphics[width=1\linewidth]{validation.png}}
        \caption{Retrieving and validation of vehicle data}
        \label{fig:placeholder}
    \end{figure}
  
  The same process applies to all string values, the \texttt{age} value only compares against the allowed range.

  Finally, the retrieved and validated values are written to the registry using a custom function. 

    \begin{figure}[h!]
        \centering
        \fbox{\includegraphics[width=1\linewidth]{write_to_registry.png}}
        \caption{Writing to the registry}
        \label{fig:placeholder}
    \end{figure}

  The figure above if illustrative as the \texttt{fprintf()} function is shortened. Each value is trimmed (spaces on the beginning and end get removed) before writing to registry.

  \item \textbf{Remove vehicle} \\
  Function first checks, if the registry is not empty. Then prompts the user to enter the index of a vehicle to be removed. Note that the program expects the user to enter index starting on 1, but the logic works zero based in the background. The program checks, if the index is valid (not negative value and within the range), and the proceeds to delete the specific line.

  My approach to deleting specific lines is based on using a temporary file. The function will start "copying" the lines to the temporary file, and once it reaches the line with received index (the line to be deleted), it skips that line.

    \begin{figure}[h!]
        \centering
        \fbox{\includegraphics[width=0.5\linewidth]{delete_line.png}}
        \caption{Deleting line in registry}
        \label{fig:placeholder}
    \end{figure}

  \item \textbf{Sort vehicles} \\
  Program first checks, if the registry is either empty or has only one vehicle. Either way, there is no need to sort. In other cases, it creates an array of \texttt{vehicle} and loads the entire registry to it. \\

  Then the function uses bubble sort algorithm to sort the vehicles inside the array. It compares two strings (owner names) using \texttt{strcmp()}, and then switches the positions respectively.

    \begin{figure}[h!]
        \centering
        \fbox{\includegraphics[width=1\linewidth]{bubble_sort.png}}
        \caption{Bubble sort algorithm}
        \label{fig:placeholder}
    \end{figure}

    Finally, it creates a temporary file in which it writes the newly sorted vehicles, and removes the old unsorted registry.

  \item \textbf{Info about vehicle} \\
  First, the program checks if the registry is not empty. Then it prompts user to enter an index, which is then validated. Information about one vehicle is then printed to the terminal.

  \item \textbf{Show all vehicles} \\
  Program checks if the registry is not empty. Next, it prompts user to enter an index, which is then validated. Information about all vehicles is then printed to the terminal.

  \item \textbf{Add random vehicle} \\
  First, program checks if the registry is not full. If not, program will then randomly choose from 5 lists of demo data and create a vehicle which then adds to the registry. The random index is generated using the \texttt{srand()} and \texttt{time()} functions:

      \begin{figure}[h!]
        \centering
        \fbox{\includegraphics[width=1\linewidth]{random.png}}
        \caption{Random selection}
        \label{fig:placeholder}
    \end{figure}

  \item \textbf{Search vehicle} \\
    First, program checks if the registry is not empty. Then prompts the user to enter a case-sensitive query - owner name.\\

    Program continues with sorting the registry, then

  \item \textbf{Quit} \\
  The program will terminate.
\end{enumerate}

\subsection*{Program structure}

In terms of modularity, I divided the program into 4 separate C files (main.c, file.c, functions.c, utils.c), one H file (file.h) and one CSV file (registry.csv):

      \begin{figure}[h!]
        \centering
        \fbox{\includegraphics[width=0.5\linewidth]{structure.png}}
        \caption{Random selection}
        \label{fig:placeholder}
    \end{figure}

    \begin{lstlisting}[language=C, caption={Definition of structs}]
    struct person {
        char name[50];
        int age;
    };
    
    struct vehicle {
        char type[20];
        char brand[20];
        char license_plate[15];
        struct person owner;
    };
    \end{lstlisting}




\end{document}